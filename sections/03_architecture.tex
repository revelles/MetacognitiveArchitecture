% placeholder for 03_architecture
The architecture decomposes cognition and governance into four layers,
L0 through L3, forming a fixed hierarchy. Each layer has a distinct
responsibility and a defined interface to the others.

\paragraph{L0: Global Cognitive Rules.}
L0 encodes non-overridable constraints on reasoning. Examples include:
(1) an abstract $\rightarrow$ domain $\rightarrow$ task reasoning flow;
(2) explicit separation of fact, inference, and unknown; and
(3) anti-fabrication discipline. Intuitively, L0 states \emph{how} all
reasoning must be structured, regardless of domain.

\paragraph{L1: Domain Reasoning Layers.}
Each domain (e.g., legal, medical, AI-systems design) receives its own
L1 layer. An L1 layer specializes the global rules to domain-specific
concepts, ontologies, and validity criteria. Domain isolation is enforced
by ensuring that L1 layers cannot directly modify L0 or other L1 layers.

\paragraph{L2: Project Governance.}
L2 operates as an orchestrator and lifecycle manager. It is responsible
for:
\begin{itemize}
    \item installing and updating L1 layers for specific projects;
    \item enforcing project isolation (separating one active project
    from another); and
    \item preserving continuity, such as ensuring that earlier
    commitments or constraints are remembered and respected.
\end{itemize}

\paragraph{L3: Testing \& Enforcement.}
L3 represents the veto and validation layer. It receives proposed
outputs together with their reasoning traces and evaluates them for
compliance with L0--L2. L3 can:
\begin{itemize}
    \item reject outputs that violate anti-fabrication rules;
    \item flag incomplete reasoning or missing conditions;
    \item enforce readiness checks before an answer is released.
\end{itemize}

\subsection*{Layer Stack Diagram}

Figure~\ref{fig:layer-stack} depicts the layer stack. The lower layers
(L0--L1) are content-producing, while the upper layers (L2--L3) are
control-governing.

\begin{figure}[h]
    \centering
    \begin{tikzpicture}[
        node distance = 0.6cm and 1.8cm,
        >=Latex,
        layer/.style={
            draw,
            rounded corners,
            thick,
            minimum width=6cm,
            minimum height=0.9cm,
            align=center
        },
        io/.style={
            draw,
            thick,
            minimum width=3.2cm,
            minimum height=0.9cm,
            align=center
        },
        arrow/.style={
            -{Latex[length=3mm]},
            thick
        },
        note/.style={
            align=left,
            font=\footnotesize
        }
    ]

    % I/O nodes
    \node[io] (input)  {User / System Instructions $\sigma$};
    \node[layer, below=of input] (L0) {L0: Global Cognitive Rules};
    \node[layer, below=of L0] (L1) {L1: Domain Reasoning Layer(s)};
    \node[layer, below=of L1] (L2) {L2: Project Governance};
    \node[layer, below=of L2] (L3) {L3: Testing \& Enforcement};
    \node[io, below=of L3] (output) {Released Output $y$};

    % Base model box on the side
    \node[layer, right=2.2cm of L1, minimum width=3.6cm] (llm) {Base Model\\$f_{\text{LLM}}$};

    % Vertical arrows
    \draw[arrow] (input)  -- (L0);
    \draw[arrow] (L0)     -- (L1);
    \draw[arrow] (L1)     -- (L2);
    \draw[arrow] (L2)     -- (L3);
    \draw[arrow] (L3)     -- (output);

    % Interaction with base model
    \draw[arrow] (L1.east) -- ++(0.8,0) |- (llm.north);
    \draw[arrow] (llm.south) |- ++(-0.8,-0.2) -| (L2.east);

    % Side annotations
    \node[note, left=1.6cm of L1] (contentnote) {Content-producing\\layers};
    \draw[dashed] (contentnote.east) -- ($(L1.west)!0.5!(L0.west)$);
    \draw[dashed] (contentnote.east) -- ($(L1.west)!0.5!(L2.west)$);

    \node[note, left=1.6cm of L3] (controlnote) {Control / governance\\layers};
    \draw[dashed] (controlnote.east) -- ($(L2.west)!0.5!(L3.west)$);

    \node[note, right=0.8cm of llm] (llmnote) {Stochastic mapping\\$\sigma \mapsto y$};

    \end{tikzpicture}
    \caption{Metacognitive Architecture as a layered pipeline. L0--L1
    transform and constrain instructions before they reach the base
    model $f_{\text{LLM}}$. L2 manages project-level governance and
    continuity, while L3 performs testing and enforcement before any
    output $y$ is released.}
    \label{fig:layer-stack}
\end{figure}

Later sections will introduce a more formal model of these layers, as
well as concrete instantiations in applied settings.
