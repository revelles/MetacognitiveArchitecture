% placeholder for 01_introduction
The recent advances in large language models (LLMs) have produced
systems capable of generating elaborate chains of reasoning on demand.
However, these systems still lack an explicit, persistent structure for
governing how reasoning is conducted over time, across domains, and
under safety or reliability constraints. Prompts can request a specific
style or discipline of reasoning, but the model does not maintain a
non-overridable hierarchy of rules that outlives any single exchange.

This paper proposes the \emph{Metacognitive Architecture \&
Instruction Layers}, a four-layer governance framework designed to act
as an Instructional Operating System for LLM-based systems. The
architecture decomposes cognition into:

\begin{itemize}
    \item L0: Global Cognitive Rules,
    \item L1: Domain Reasoning Layers,
    \item L2: Project Governance, and
    \item L3: Testing \& Enforcement.
\end{itemize}

The core idea is to separate \emph{content} (what the model says) from
\emph{control} (how and under which constraints it is allowed to say it).
This decoupling enables structured continuity, explicit anti-fabrication
checks, and isolation between domains such as legal reasoning, clinical
support, and systems design.

\paragraph{Contributions.} The main contributions of this work are:
\begin{enumerate}[label=(\arabic*)]
    \item a conceptual and formal definition of a four-layer
    metacognitive architecture for LLM-governed reasoning systems;

    \item a control/content decoupling scheme in which L0--L1 define
    reasoning content and L2--L3 define governance and enforcement;

    \item worked examples illustrating how the architecture can be
    instantiated in practical domains requiring continuity, procedural
    rigor, and cross-domain isolation.
\end{enumerate}

The remainder of this paper is organized as follows. Section~\ref{sec:background}
situates the work within related efforts in tool-augmented LLMs,
guardrails, and system prompts. Section~\ref{sec:architecture} introduces
the four layers and their interactions. Section~\ref{sec:formal-model}
presents a formalization of the architecture. Section~\ref{sec:examples}
gives worked examples, and Section~\ref{sec:related-work} discusses
connections to prior work. We conclude in Section~\ref{sec:conclusion}.
As a working reference, we denote this framework as
\emph{Metacognitive Architecture \& Instruction Layers}~\cite{revelles2025metacognitive}.
