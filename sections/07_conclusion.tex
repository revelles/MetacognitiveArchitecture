% placeholder for 07_conclusion
This paper introduced the Metacognitive Architecture \& Instruction
Layers, a four-layer governance framework designed to act as an
Instructional Operating System for LLM-based reasoning systems. By
separating content-producing layers (L0--L1) from control-governing
layers (L2--L3), the architecture creates space for continuity,
cross-domain isolation, and explicit anti-fabrication enforcement.

Several directions remain for future work. On the theoretical side,
richer formal models could clarify the contracts between layers and
enable partial verification. On the practical side, implementing the
architecture in real-world systems---and stress-testing it on
high-stakes tasks---will be essential for understanding its strengths
and limitations.

Ultimately, the goal is not to freeze a single configuration of rules,
but to provide a reusable, extensible scaffolding for metacognitive
control. As LLMs become more capable and are embedded in safety-critical
workflows, explicit, layered governance over their reasoning will become
not just desirable, but necessary.
