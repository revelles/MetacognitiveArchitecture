% placeholder for 06_related_work
The proposed architecture intersects multiple research threads, including
tool-augmented LLMs, AI safety and alignment, program synthesis and
verification, and cognitive architectures.

\paragraph{Tool-augmented and agentic LLMs.}
Work on tool use and agents gives LLMs the ability to call external
functions, maintain working memory, or coordinate multiple roles.
However, these frameworks often treat high-level governance as an
implementation detail rather than a first-class object. The present
architecture can be viewed as a way to factor such systems into
explicit, named layers with clearly articulated responsibilities.

\paragraph{Safety and guardrails.}
Guardrail systems define policies about allowed content and sometimes
about required checks. Our L3 layer plays a complementary role: it
provides a structural place where policy enforcement, readiness checks,
and verification can be composed and reasoned about.

\paragraph{Cognitive architectures.}
Classical cognitive architectures and metacognition research study how
systems can monitor and regulate their own reasoning processes. The
Metacognitive Architecture \& Instruction Layers reinterprets these
ideas in the specific context of LLM-governed systems, emphasizing the
need for persistent, hierarchical instruction governance.

A more detailed comparison with specific systems and frameworks can be
developed as the architecture is instantiated in concrete
implementations.
