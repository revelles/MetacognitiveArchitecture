\documentclass[12pt]{article}

% --- Page layout ---
\usepackage[margin=1in]{geometry}

% --- Math and theorem tools ---
\usepackage{amsmath, amssymb, amsthm}
\usepackage{mathtools}

% --- Figures, tables, lists ---
\usepackage{graphicx}
\usepackage{booktabs}
\usepackage{enumitem}

% --- Hyperlinks ---
\usepackage[hidelinks]{hyperref}

% --- TikZ for diagrams ---
\usepackage{tikz}
\usetikzlibrary{arrows.meta, positioning}

% --- Theorem-like environments ---
\newtheorem{definition}{Definition}
\newtheorem{theorem}{Theorem}
\newtheorem{lemma}{Lemma}

% --- Metadata ---
\title{Metacognitive Architecture \& Instruction Layers:\\
A Modular Governance Framework for Reasoning Systems}
\author{Francisco Revelles}
\date{\today}

\begin{document}

\maketitle

\begin{abstract}
Large language models demonstrate increasingly sophisticated reasoning
but lack persistent, hierarchical governance over cognition, continuity,
and anti-fabrication safeguards. This paper introduces the
\emph{Metacognitive Architecture \& Instruction Layers}, a modular
framework designed to implement an Instructional Operating System (IOS)
for AI systems. By decoupling content-production layers (L0--L1) from
control-governance layers (L2--L3), the architecture enables structured,
auditable, and domain-isolated reasoning suitable for high-stakes,
long-horizon applications.
\end{abstract}

\section{Introduction}\label{sec:introduction}
% placeholder for 01_introduction
The recent advances in large language models (LLMs) have produced
systems capable of generating elaborate chains of reasoning on demand.
However, these systems still lack an explicit, persistent structure for
governing how reasoning is conducted over time, across domains, and
under safety or reliability constraints. Prompts can request a specific
style or discipline of reasoning, but the model does not maintain a
non-overridable hierarchy of rules that outlives any single exchange.

This paper proposes the \emph{Metacognitive Architecture \&
Instruction Layers}, a four-layer governance framework designed to act
as an Instructional Operating System for LLM-based systems. The
architecture decomposes cognition into:

\begin{itemize}
    \item L0: Global Cognitive Rules,
    \item L1: Domain Reasoning Layers,
    \item L2: Project Governance, and
    \item L3: Testing \& Enforcement.
\end{itemize}

The core idea is to separate \emph{content} (what the model says) from
\emph{control} (how and under which constraints it is allowed to say it).
This decoupling enables structured continuity, explicit anti-fabrication
checks, and isolation between domains such as legal reasoning, clinical
support, and systems design.

\paragraph{Contributions.} The main contributions of this work are:
\begin{enumerate}[label=(\arabic*)]
    \item a conceptual and formal definition of a four-layer
    metacognitive architecture for LLM-governed reasoning systems;

    \item a control/content decoupling scheme in which L0--L1 define
    reasoning content and L2--L3 define governance and enforcement;

    \item worked examples illustrating how the architecture can be
    instantiated in practical domains requiring continuity, procedural
    rigor, and cross-domain isolation.
\end{enumerate}

The remainder of this paper is organized as follows. Section~\ref{sec:background}
situates the work within related efforts in tool-augmented LLMs,
guardrails, and system prompts. Section~\ref{sec:architecture} introduces
the four layers and their interactions. Section~\ref{sec:formal-model}
presents a formalization of the architecture. Section~\ref{sec:examples}
gives worked examples, and Section~\ref{sec:related-work} discusses
connections to prior work. We conclude in Section~\ref{sec:conclusion}.
As a working reference, we denote this framework as
\emph{Metacognitive Architecture \& Instruction Layers}~\cite{revelles2025metacognitive}.


\section{Background and Motivation}\label{sec:background}
% placeholder for 02_background
This section situates the proposed architecture relative to existing
approaches for controlling and structuring LLM behavior. We highlight
three threads: (i) prompt engineering and system prompts, (ii) tool and
agent frameworks, and (iii) safety, guardrail, and verification work.

\paragraph{Prompt engineering and system prompts.}
Much of the practical control over LLM behavior today is expressed
through long system prompts, templates, and few-shot examples. These
techniques can encode style, constraints, or domain-specific rules, but
they lack persistent structure: there is no native separation between
global rules and per-project instructions, and no explicit notion of
testing and enforcement layers.

\paragraph{Tools and multi-agent frameworks.}
Agent and tool frameworks introduce explicit roles, enabling models to
call external tools (search, code execution, retrieval) and to subdivide
tasks across multiple agents. While these frameworks introduce a form of
structure, they typically do not formalize a layered governance model
over cognition itself. The question of \emph{who} enforces global rules
and \emph{how} cross-domain isolation is guaranteed remains loosely
specified.

\paragraph{Safety, guardrails, and verification.}
Guardrail systems and safety policies attempt to prevent models from
emitting harmful or disallowed content. Verification-oriented work
checks individual answers or proofs for correctness. These directions
address important aspects of risk, but usually at the level of content
filtering or post-hoc checking, rather than a general metacognitive
architecture that governs how reasoning is structured from the outset.

The Metacognitive Architecture \& Instruction Layers proposed in this
paper can be seen as a unifying abstraction over these threads. It
organizes the control space into a fixed hierarchy of layers (L0--L3)
that can host prompts, tools, and guardrails in a principled manner.


\section{The Layered Architecture}\label{sec:architecture}
% placeholder for 03_architecture
The architecture decomposes cognition and governance into four layers,
L0 through L3, forming a fixed hierarchy. Each layer has a distinct
responsibility and a defined interface to the others.

\paragraph{L0: Global Cognitive Rules.}
L0 encodes non-overridable constraints on reasoning. Examples include:
(1) an abstract $\rightarrow$ domain $\rightarrow$ task reasoning flow;
(2) explicit separation of fact, inference, and unknown; and
(3) anti-fabrication discipline. Intuitively, L0 states \emph{how} all
reasoning must be structured, regardless of domain.

\paragraph{L1: Domain Reasoning Layers.}
Each domain (e.g., legal, medical, AI-systems design) receives its own
L1 layer. An L1 layer specializes the global rules to domain-specific
concepts, ontologies, and validity criteria. Domain isolation is enforced
by ensuring that L1 layers cannot directly modify L0 or other L1 layers.

\paragraph{L2: Project Governance.}
L2 operates as an orchestrator and lifecycle manager. It is responsible
for:
\begin{itemize}
    \item installing and updating L1 layers for specific projects;
    \item enforcing project isolation (separating one active project
    from another); and
    \item preserving continuity, such as ensuring that earlier
    commitments or constraints are remembered and respected.
\end{itemize}

\paragraph{L3: Testing \& Enforcement.}
L3 represents the veto and validation layer. It receives proposed
outputs together with their reasoning traces and evaluates them for
compliance with L0--L2. L3 can:
\begin{itemize}
    \item reject outputs that violate anti-fabrication rules;
    \item flag incomplete reasoning or missing conditions;
    \item enforce readiness checks before an answer is released.
\end{itemize}

\subsection*{Layer Stack Diagram}

Figure~\ref{fig:layer-stack} depicts the layer stack. The lower layers
(L0--L1) are content-producing, while the upper layers (L2--L3) are
control-governing.

\begin{figure}[h]
    \centering
    \begin{tikzpicture}[
        node distance = 0.6cm and 1.8cm,
        >=Latex,
        layer/.style={
            draw,
            rounded corners,
            thick,
            minimum width=6cm,
            minimum height=0.9cm,
            align=center
        },
        io/.style={
            draw,
            thick,
            minimum width=3.2cm,
            minimum height=0.9cm,
            align=center
        },
        arrow/.style={
            -{Latex[length=3mm]},
            thick
        },
        note/.style={
            align=left,
            font=\footnotesize
        }
    ]

    % I/O nodes
    \node[io] (input)  {User / System Instructions $\sigma$};
    \node[layer, below=of input] (L0) {L0: Global Cognitive Rules};
    \node[layer, below=of L0] (L1) {L1: Domain Reasoning Layer(s)};
    \node[layer, below=of L1] (L2) {L2: Project Governance};
    \node[layer, below=of L2] (L3) {L3: Testing \& Enforcement};
    \node[io, below=of L3] (output) {Released Output $y$};

    % Base model box on the side
    \node[layer, right=2.2cm of L1, minimum width=3.6cm] (llm) {Base Model\\$f_{\text{LLM}}$};

    % Vertical arrows
    \draw[arrow] (input)  -- (L0);
    \draw[arrow] (L0)     -- (L1);
    \draw[arrow] (L1)     -- (L2);
    \draw[arrow] (L2)     -- (L3);
    \draw[arrow] (L3)     -- (output);

    % Interaction with base model
    \draw[arrow] (L1.east) -- ++(0.8,0) |- (llm.north);
    \draw[arrow] (llm.south) |- ++(-0.8,-0.2) -| (L2.east);

    % Side annotations
    \node[note, left=1.6cm of L1] (contentnote) {Content-producing\\layers};
    \draw[dashed] (contentnote.east) -- ($(L1.west)!0.5!(L0.west)$);
    \draw[dashed] (contentnote.east) -- ($(L1.west)!0.5!(L2.west)$);

    \node[note, left=1.6cm of L3] (controlnote) {Control / governance\\layers};
    \draw[dashed] (controlnote.east) -- ($(L2.west)!0.5!(L3.west)$);

    \node[note, right=0.8cm of llm] (llmnote) {Stochastic mapping\\$\sigma \mapsto y$};

    \end{tikzpicture}
    \caption{Metacognitive Architecture as a layered pipeline. L0--L1
    transform and constrain instructions before they reach the base
    model $f_{\text{LLM}}$. L2 manages project-level governance and
    continuity, while L3 performs testing and enforcement before any
    output $y$ is released.}
    \label{fig:layer-stack}
\end{figure}

Later sections will introduce a more formal model of these layers, as
well as concrete instantiations in applied settings.


\section{Formal Model}\label{sec:formal-model}
This section presents a more structured formalization of the
metacognitive architecture. The goal is not to fix a single canonical
formalism, but to show that L0--L3 can be modeled as operators over
instructions, states, and outputs, with clearly separated roles for
content and control.

\subsection*{Basic objects}

Let:
\begin{itemize}
    \item $\Sigma$ be the set of possible instructions (prompts,
    policies, context strings),
    \item $\mathcal{O}$ be the set of possible outputs, and
    \item $\mathcal{S}$ be the space of internal states of the system
    (including project history, active domains, and memory).
\end{itemize}

A bare language model can be idealized as a (possibly stochastic) map
\[
    f_{\text{LLM}} : \Sigma \times \mathcal{S}
      \to \mathcal{O} \times \mathcal{S}.
\]
We make no assumption here about how $f_{\text{LLM}}$ is implemented; it
could be any autoregressive or retrieval-augmented architecture.

\subsection*{Layer operators}

The metacognitive architecture introduces four classes of operators,
$L_0, L_1, L_2, L_3$, each acting on $(\sigma, s)$ or on candidate
outputs.

\begin{definition}[Global cognitive layer $L_0$]
The L0 layer is an operator
\[
    L_0 : \Sigma \times \mathcal{S} \to \Sigma \times \mathcal{S}
\]
that enforces global reasoning constraints. Typical effects include:
normalizing the instruction into an abstract $\rightarrow$ domain
$\rightarrow$ task structure, appending global anti-fabrication rules,
or imposing a fact / inference / unknown separation discipline.
\end{definition}

\begin{definition}[Domain reasoning layers $L_1^d$]
For each domain $d$ in some index set $\mathcal{D}$ (e.g., legal,
medical, AI-systems), there is a domain-specific layer
\[
    L_1^d : \Sigma \times \mathcal{S} \to \Sigma \times \mathcal{S}
\]
that specializes instructions to domain $d$, injecting ontologies,
local constraints, and validity conditions. A configuration may enable a
subset $\mathcal{D}_{\text{active}} \subseteq \mathcal{D}$.
\end{definition}

\begin{definition}[Project governance layer $L_2$]
The L2 layer manages project-level state. Given a project identifier
$p$ from some set of projects $\mathcal{P}$, we write
\[
    L_2^p : \Sigma \times \mathcal{S} \times \mathcal{D}_{\text{active}}
      \to \Sigma \times \mathcal{S}.
\]
Intuitively, $L_2^p$:
\begin{enumerate}[label=(\alph*)]
    \item selects which domain layers $L_1^d$ are active for project $p$,
    \item re-applies project constraints (e.g., prior commitments,
          safety requirements),
    \item maintains continuity by updating the project state component
          of $\mathcal{S}$.
\end{enumerate}
\end{definition}

\begin{definition}[Testing and enforcement layer $L_3$]
Given a candidate output $y \in \mathcal{O}$, an associated reasoning
trace $\tau$, and state $s \in \mathcal{S}$, the L3 layer produces an
enforcement decision
\[
    L_3 : \mathcal{O} \times \mathcal{T} \times \mathcal{S}
      \to \{ \text{accept}, \text{revise}, \text{reject} \}
\]
for some space of traces $\mathcal{T}$. In the case $\text{revise}$,
$L_3$ may also emit a refined instruction $\sigma_{\text{rev}}$ to be
fed back into the pipeline.
\end{definition}

\subsection*{Execution pipeline}

A single reasoning episode under project $p \in \mathcal{P}$ proceeds as
follows. Given an initial instruction $\sigma_0 \in \Sigma$ and state
$s_0 \in \mathcal{S}$:

\begin{enumerate}[label=(\roman*)]
    \item \textbf{Global normalization (L0).}
    \[
        (\sigma_1, s_1) = L_0(\sigma_0, s_0).
    \]

    \item \textbf{Domain specialization (L1).}
    Let $\mathcal{D}_{\text{active}}$ be the set of domain layers
    chosen for project $p$. We apply them in some admissible order
    (e.g., a fixed total order or a partial order resolved by $L_2$):
    \[
        (\sigma_2, s_2)
           = \Big( \prod_{d \in \mathcal{D}_{\text{active}}} L_1^d \Big)
             (\sigma_1, s_1).
    \]

    \item \textbf{Project governance (L2).}
    \[
        (\sigma_3, s_3)
           = L_2^p(\sigma_2, s_2, \mathcal{D}_{\text{active}}).
    \]

    \item \textbf{Base model call.}
    \[
        (y, s_4) = f_{\text{LLM}}(\sigma_3, s_3).
    \]
    During this step, a reasoning trace $\tau$ is also accumulated
    (e.g., chain-of-thought, tool calls, or intermediate states).

    \item \textbf{Testing and enforcement (L3).}
    L3 evaluates $(y, \tau, s_4)$:
    \[
        d = L_3(y, \tau, s_4).
    \]
    If $d = \text{accept}$, the system releases $y$. If $d =
    \text{reject}$, the output is withheld. If $d = \text{revise}$,
    the system obtains a revised instruction $\sigma_{\text{rev}}$ and
    may re-enter the pipeline at step (i) or (ii).
\end{enumerate}

We call this composition the \emph{governed reasoning pipeline} for
project $p$.

\subsection*{Separation of content and control}

The architecture prescribes a structural separation:

\begin{itemize}
    \item Content-producing components are $f_{\text{LLM}}$ together
    with the instruction transformations in L0 and L1.
    \item Control-governing components are $L_2$ and $L_3$, which
    decide which domains are active, how project state evolves, and
    whether candidate outputs are acceptable.
\end{itemize}

This separation can be made precise by viewing $L_2$ and $L_3$ as
operating on a \emph{meta-level} configuration that is not writable by
$L_1^d$ or $f_{\text{LLM}}$ directly. One can then reason about
invariants such as:

\begin{itemize}
    \item domain isolation (no project may activate conflicting domain
          rules simultaneously);
    \item anti-fabrication discipline (L3 rejects outputs that violate
          L0's fact / inference / unknown separation);
    \item continuity (L2 re-applies project constraints across
          episodes).
\end{itemize}

These invariants are enforced by construction at the level of the
pipeline, rather than as ad-hoc prompt instructions. A more detailed
formal treatment could use transition systems or modal logics to reason
about reachable states and safety properties, but the operator view
above is sufficient to capture the intended governance semantics.


\section{Worked Examples and Applications}\label{sec:examples}
% placeholder for 05_examples
To illustrate the architecture, we outline several scenarios in which
L0--L3 provide governance that is difficult to obtain from prompts
alone. The examples are intentionally heterogeneous to emphasize
cross-domain isolation and continuity.

\paragraph{Example 1: Legal reasoning vs.\ medical reasoning.}
In one project, an L1 layer is instantiated for legal reasoning under a
specific jurisdiction. In a separate project, another L1 layer is used
for clinical communication. L2 ensures that domain-specific rules (e.g.,
evidentiary standards) are not silently imported into clinical
explanations, and vice versa. L3 enforces that any legal answer carries
an explicit separation between fact, inference, and unknown.

\paragraph{Example 2: Long-horizon project continuity.}
Consider a multi-week technical project where an AI assistant helps
refine a system design. L2 maintains project-level constraints (such as
a chosen architecture style or safety requirement) and re-applies them
across sessions. L3 rejects outputs that regress on previously agreed
constraints, forcing the system to either justify deviations or stay
consistent.

\paragraph{Example 3: Anti-fabrication discipline.}
In a research or compliance setting, L0 imposes a strict
fact/inference/unknown separation. When the base LLM proposes an answer,
L3 inspects the trace for unsupported claims. If unresolved gaps remain,
L3 can either request additional evidence (via tools) or downgrade the
answer to an explicitly uncertain status, rather than silently
hallucinating details.

These examples are not exhaustive, but they demonstrate how a layered
metacognitive architecture can turn informal prompting patterns into an
explicit, auditable control structure.


\section{Related Work}\label{sec:related-work}
% placeholder for 06_related_work
The proposed architecture intersects multiple research threads, including
tool-augmented LLMs, AI safety and alignment, program synthesis and
verification, and cognitive architectures.

\paragraph{Tool-augmented and agentic LLMs.}
Work on tool use and agents gives LLMs the ability to call external
functions, maintain working memory, or coordinate multiple roles.
However, these frameworks often treat high-level governance as an
implementation detail rather than a first-class object. The present
architecture can be viewed as a way to factor such systems into
explicit, named layers with clearly articulated responsibilities.

\paragraph{Safety and guardrails.}
Guardrail systems define policies about allowed content and sometimes
about required checks. Our L3 layer plays a complementary role: it
provides a structural place where policy enforcement, readiness checks,
and verification can be composed and reasoned about.

\paragraph{Cognitive architectures.}
Classical cognitive architectures and metacognition research study how
systems can monitor and regulate their own reasoning processes. The
Metacognitive Architecture \& Instruction Layers reinterprets these
ideas in the specific context of LLM-governed systems, emphasizing the
need for persistent, hierarchical instruction governance.

A more detailed comparison with specific systems and frameworks can be
developed as the architecture is instantiated in concrete
implementations.


\section{Conclusion}\label{sec:conclusion}
% placeholder for 07_conclusion
This paper introduced the Metacognitive Architecture \& Instruction
Layers, a four-layer governance framework designed to act as an
Instructional Operating System for LLM-based reasoning systems. By
separating content-producing layers (L0--L1) from control-governing
layers (L2--L3), the architecture creates space for continuity,
cross-domain isolation, and explicit anti-fabrication enforcement.

Several directions remain for future work. On the theoretical side,
richer formal models could clarify the contracts between layers and
enable partial verification. On the practical side, implementing the
architecture in real-world systems---and stress-testing it on
high-stakes tasks---will be essential for understanding its strengths
and limitations.

Ultimately, the goal is not to freeze a single configuration of rules,
but to provide a reusable, extensible scaffolding for metacognitive
control. As LLMs become more capable and are embedded in safety-critical
workflows, explicit, layered governance over their reasoning will become
not just desirable, but necessary.


\bibliographystyle{plain}
\bibliography{refs}

\end{document}